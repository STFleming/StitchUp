\lstset(language=C)
\begin{lstlisting}[caption="Matrix Multiplication Example", label=lst:MMM]
void MMM(float A[5][5], B[5][5], C[5][5])
{
	int i=0, j=0, k=0;
	for(i=0, i<5, i++)
	{
		C[i][j] = 0;
		for(j=0, j<5; j++)
		{
			C[i][j] += A[i][k]*B[k][j];
		}
	} 
}
\end{lstlisting}

\subsection{Matrix Multiplication Example}
To demonstrate and highlight the potential saving that duplicating just
the control flow shadow can achieve we will use the classic example of
matrix multiplication, as seen in Listing  


\subsection{Threat Model}
What is the threat model?
Use some Set notation to make what is being protected more explicit.

What can we gaurantee with this threat model, and what sort of architecture
do we need to define to do this?

How can this be practically used in the real world?

