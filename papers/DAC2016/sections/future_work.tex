This will mainly be conclusion.

\subsection{Memory Analysis}
Currently StitchUp performs no analysis on memory regions of the code,
where if a single control set instruction writes to a memory location
then any read from that location is assumed to also be control dependant.
Through Analysing the memory access pattern using separation logic, it should
be possible to segregate out the control dependant memory from the rest.
Not will this act as an optimisation potentially reducing the number of
instructions that are considered part of the control set, but it will also
open up other interesting research directions. For example, it might be possible
for us to reduce the reliability of the non control memory through adjusting
the refresh rate of the DRAM cells, such as in \cite{liu2012flikker}.
Or it might be possible to investigate selectively applying ECC checks to
only control dependant memory regions.
